\let\negmedspace\undefined
\let\negthickspace\undefined
\documentclass[journal]{IEEEtran}
\usepackage[a5paper, margin=10mm, onecolumn]{geometry}
%\usepackage{lmodern} % Ensure lmodern is loaded for pdflatex
\usepackage{tfrupee} % Include tfrupee package

\setlength{\headheight}{1cm} % Set the height of the header box
\setlength{\headsep}{0mm}     % Set the distance between the header box and the top of the text

\usepackage{gvv-book}
\usepackage{gvv}
\usepackage{cite}
\usepackage{amsmath,amssymb,amsfonts,amsthm}
\usepackage{algorithmic}
\usepackage{graphicx}
\usepackage{textcomp}
\usepackage{xcolor}
\usepackage{txfonts}
\usepackage{listings}
\usepackage{enumitem}
\usepackage{mathtools}
\usepackage{gensymb}
\usepackage{comment}
\usepackage[breaklinks=true]{hyperref}
\usepackage{tkz-euclide} 
\usepackage{listings}
% \usepackage{gvv}                                        
\def\inputGnumericTable{}                                 
\usepackage[latin1]{inputenc}                                
\usepackage{color}                                            
\usepackage{array}                                            
\usepackage{longtable}                                       
\usepackage{calc}  
\usepackage{amsmath,amssymb}

\usepackage{multirow}                                         
\usepackage{hhline}                                           
\usepackage{ifthen}                                           
\usepackage{lscape}
\begin{document}

\bibliographystyle{IEEEtran}

\title{
%	\logo{
JEE MAINS

\large{EE1030}

MARCH 17 - SHIFT - 2
%	}
}
\author{Homa Harshitha Vuddanti

(EE24BTECH11062)
}	

\maketitle

\bigskip

\renewcommand{\thefigure}{\theenumi}
\renewcommand{\thetable}{\theenumi}
QUESTIONS- 16 TO 30\\
SECTION A
\begin{enumerate}
   
\item If the sides $AB, BC$, and $CA$ of a triangle $ABC$ have, 3, 5 and 6 interior points respectively, then the total number of triangles that can be constructed using these points as vertices is equal to:
\begin{enumerate}
    \item 360
    \item 240
    \item 333
    \item 364
\end{enumerate}
 \item The value of $\lim_{x \to \infty}\sbrak{r}+\sbrak{2r}+\dots\sbrak{nr}/\sbrak{n^2}$, where r is a non-zero number and \sbrak{r} denotes the greatest integer less than or equal to $r$, is equal to :
 \begin{enumerate}
     \item 0
     \item $r$
     \item $r/2$
     \item $2r$
 \end{enumerate}
 
 \item The value of $\sum_{r=0}^{6}\brak{^6C_r.^6C_{6-r}}$ is equal to :
 \begin{enumerate}
     \item 1124
     \item 924
     \item 1324
     \item 1024
 \end{enumerate}
 
\item Two tangents are drawn from a point $\vec{P}$ to the circle $x^2 + y^2 -2x-4y+4=0$, such that the angle between these tangents is  $\tan^{-1}\brak{12/5}$, where $\tan^{-1}\brak{12/5}\in \brak{0,\pi}$. If the centre of the circle is denoted by $\vec{C}$ and these tangents touch the circle at points $\vec{A}$ and $\vec{B}$, then the ratio of the areas of $\triangle PAB$ and $\triangle CAB$ is:
\begin{enumerate}
    \item 11:4
    \item 9:4
    \item 2:1
    \item 3:1
\end{enumerate}

\item The number of solutions of the equation $\sin^{-1}\sbrak{x^2+\brak{1/3}}+\cos^{-1}\sbrak{x^2-\brak{2/3}}=x^2$, for $x \in \sbrak{-1,1}$, and \sbrak{x} denotes the greatest integer less than or equal to $x$, is:
\begin{enumerate}
    \item 0
    \item 2
    \item 4
    \item infinite
\end{enumerate}
SECTION B
\item Let the coefficients of third, fourth and fifth terms in the expansion of $\sbrak{x+\brak{a/x^2}}^n, x\neq 0$ be in the ratio 12:8:3. Then the term independent of $x$ in the expansion is equal to 

\item Let $A=\begin{bmatrix}a & b \\ c& d\end{bmatrix}$ and $B=\begin{bmatrix}\alpha \\ \beta\end{bmatrix}\neq \begin{bmatrix}0\\0\end{bmatrix}$ such that $AB=B$ and $a+d=2021$, then the value of $ad-bc$ is equal to 

\item Let $f: \sbrak{-1,1}\mapsto R$ be defined as $f\brak{x}=ax^2 +bx+c$ for all $x\in \sbrak{-1,1}$, where $a,b,c \in R$ such that $f\brak{-1}=2, f^{\prime}\brak{-1}=1$ and for $x\in \sbrak{-1,1}$ the maximum value of $f^{\prime\prime}\brak{x}$ is 1/2. If $f\brak{x}\leq \alpha, x\in \sbrak{-1,1},$ then least value of $\alpha$ is equal to

\item Let $I_n = 
\int_{1}^{e} x^{19}\brak{log\abs{x}}^n \, dx$, where $n \in N$. If $20\brak{I_{10}}=\alpha I_9+\beta I_8$, for natural numbers $\alpha$ and $\beta$, then $\alpha-\beta$ is equal to

\item Let $f:\sbrak{-3,1}\mapsto R$ be given as $f\brak{x}=$ $\begin{cases}
min\sbrak{\brak{x+6,x^2}}, -3\leq x \leq 0\\
max\sbrak{\sqrt{x}, x^2}, 0\leq x \leq 1
\end{cases}$
If the area bounded by $y=f\brak{x}$ and x-axis is $A$, then the value of $6A$ is equal to
\item Let $\vec{x}$ be a vector in the plane containing vectors $a=2i-j+k$ and $b=i+2j-k$. If the vector $\vec{x}$ is perpendicular to \brak{3i+2j-k} and its projection on $a$ is $17\sqrt{6}/2$, then the value of $x^2$ is equal to 
\item Consider a set of $3n$ numbers having variance 4. In this set, the mean of the first $2n$ numbers is 6 and the mean of the remaining $n$ numbers is 3. A new set is constructed by adding 1 into each of the first $2n$ numbers and subtracting 1 from each of the remaining $n$ numbers. If the variance of the new set is $k$, then $9k$ is equal to 
\item If 1, $log_{10}\brak{4^x-2}$ and $log_{10}\brak{4^x+\brak{18/5}}$ are in arithmetic progression for a real number $x$, then the value of the determinant $\begin{bmatrix}2\sbrak{x-\brak{1/2}}& x-1 & x^2 \\ 1 & 0 & x \\ x & 1 & 0\end{bmatrix}$ is equal to 
\item Let $\vec{P}$ be an arbitary point having the sum of squares of the distances from the planes $x+y+z=0, lx-nz=0$ and $x-2y+z=0$, equal to 9. If the locus of the point $\vec{P}$ is $x^2+y^2+z^2=9$ then the value of $l-n$ is equal to 
\item Let $\tan \alpha, \tan \beta$ and $\tan \gamma;\alpha,\beta,\gamma\neq\sbrak{2n-1}\pi/2,n\in N$ be the slopes of three-line segment $OA,OB$ and $OC$, respectively, where $\vec{O}$ is origin. If the circumcentre of triangle $ABC$ coincides with the origin and its orthocentre lies on y-axis, then the value of $\sbrak{\brak{\cos 3\alpha +\cos 3\beta +\cos 3\gamma}/\brak{\cos \alpha *\cos \beta *\cos \gamma}}^2$ is equal to 
\end{enumerate}
\end{document}
